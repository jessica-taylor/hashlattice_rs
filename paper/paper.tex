
\documentclass{article}

\usepackage{amsthm}
\usepackage{amsmath}
\usepackage{amssymb}
\newtheorem{theorem}{Theorem}
\newtheorem{lemma}{Lemma}

\title{Semilattice Graphs for Distributed Databases}
\author{Jessica Taylor}
\date{\today}

\begin{document}
    \maketitle
    
    \section{Introduction}
        A \emph{semilattice} is a set $L$ with a binary operation $\vee$ (``join'') that is commutative ($x \vee y = y \vee x$), associative ($(x \vee y) \vee z = x \vee (y \vee z)$), and idempotent ($x \vee x = x$).

        We write $x \leq y$ to mean $y = x \vee y$. Note that if $x \leq y$ and $y \leq x$, then $x = x \vee y = y$. Additionally, $x \leq x$, and if $x \leq y$ and $y \leq z$, then $z = z \vee y = z \vee y \vee x \geq x$. Accordingly, $\leq$ forms a partial order.

        In databases, lattices are useful for defining a notion of a most updated value. For example, with $L$ as a set of sets of messages and $\vee = \cup$, the lattice represents sets of messages that are updated to include more messages as they come in.

    \section{Semilattices as a category}

      We discuss category theoretic language for semilattices to derive a notion of dependency between semilattices. In the next section, we drop the category-theoretic language, so it is not strictly necessary to understand category theory to understand the result, although it helps with the derivation.

      A semilattice can be interpreted as a category whose objects are the members of $L$, and where a morphism exists between $x$ and $y$ if and only if $x \leq y$; in this case there must be exactly one morphism of this form denoted $x \leq y$. To be a semilattice, the category must have coproducts (corresponding to joins).

      Semilattices form a category $\mathrm{Semilattice}$.  The objects are semilattices and the morphisms are join-preserving functors between semilattices. A functor between two semilattices $L_1, L_2$ is, by the definition of functor, a function $f: L_1 \rightarrow L_2$ mapping elements of $L_1$ to elements of $L_2$, such that if $x \leq y$, then $f(x) \leq f(y)$. The functor preserves joins if for $x_1 : L_1, x_2 : L_2$, $f(x_1 \vee x_2) = f(x_1) \vee f(x_2)$. The $\mathrm{Semilattice}$ category has products in the obvious way.

      Note that any join-preserving function $f : L_1 \rightarrow L_2$ is automatically a functor: if $x \leq y$, then $f(x) \leq f(x) \vee f(y) = f(x \vee y) = f(y)$. 

      A semilattice fibration on a semilattice $L$ is a functor $g : L \rightarrow \mathrm{Semilattice}$, which assigns to each element $x$ of $L$ a semilattice $g(x)$, and assigns to each ordering $x \leq y$ a semilattice morphism (i.e. join-preserving functor) $g(x \leq y)$ from $g(x)$ to $g(y)$.

      The total space $\mathrm{SemiLTot}(L, g)$ is a preorder whose elements consist of pairs $(x, x')$ where $x : L$ and $x' : g(x)$. We consider $(x, x') \leq (y, y')$ when $x \leq y$ and $g(x \leq y)(x') \leq y'$. Consider two elements of the total space, $(x, x'), (y, y')$. Their join is defined as $$(x, x') \vee (y, y') := (x \vee y', g(x \leq x \vee y)(x') \vee g(y \leq x \vee y)(y'))$$. 

      \begin{theorem}
        The total space $\mathrm{SemiLTot}(L, g)$ of a semilattice $L$ and a semilattice fibration $g$ is a semilattice.
      \end{theorem}
      \begin{proof}
        We must show that this join is commutative, associative, and idempotent.
        \begin{itemize}
          \item \emph{Commutative}:
            \begin{align*}
              (x, x') \vee (y, y') &= \left(x \vee y, g(x \leq x \vee y)(x') \vee g(y \leq x \vee y)(y')\right) \\
              &= \left(y \vee x, g(y \leq y \vee x)(y') \vee g(x \leq y \vee x)(x')\right) \\
              &= (y, y') \vee (x, x')
            \end{align*}
          \item \emph{Associative}:
            \begin{align*} &((x, x') \vee (y, y')) \vee (z, z')  \\
              =& \left(x \vee y \vee z, g(x \vee y \leq x \vee y \vee z)(g(x \leq x \vee y)(x') \vee g(y \leq x \vee y)(y')) \vee g(z \leq x \vee y \vee z)(z')\right) \\
              =& \left(x \vee y \vee z, g(x \leq x \vee y \vee z)(x') \vee g(y \leq x \vee y \vee z)(y') \vee g(z \leq x \vee y \vee z)(z')\right) \\
              =& \left(x \vee y \vee z, g(x \leq x \vee y \vee z)(x') \vee g(y \vee z \leq x \vee y \vee z)(g(y \leq y \vee z)(y') \vee g(z \leq y \vee z)(z'))\right) \\
              =& (x, x') \vee ((y, y') \vee (z, z'))
            \end{align*}
          \item \emph{Idempotent}: 
            \begin{align*}
              (x, x') \vee (x, x') &= \left(x \vee x, g(x \leq x \vee x)(x') \vee g(x \leq x \vee x)(x')\right) \\
              &= \left(x, g(x \leq x)(x')\right) \\
              &= (x, x')
            \end{align*}
        \end{itemize}
      \end{proof}

      Note that the projection $\pi_1 : \mathrm{Tot}(L, g) \rightarrow L$ is a semilattice morphism.

    \section{De-categorizing semilattice fibrations}

      To define a semilattice fibration more directly, we can remove the category-theoretic language, replacing category-theoretic entities with their definitions.

      A semilattice fibration consists of a function $l : L \rightarrow \mathrm{Semilattice}$, assigning to each element of $L$ a semilattice, and a function $t : \Pi (x, y : L, x \leq y) l(x) \rightarrow l(y)$ (a ``transport function''), satisfying:

      \begin{itemize}
        \item For any $x : L, x' : l(x)$: $$t(x, x) = \mathsf{id}$$.
        \item For any $x, y, z : L$ such that $x \leq y \leq z$, and $x' : l(x)$: $$t(y, z) \circ t(x, y) = t(x, z)$$.
        \item For any $x, y : L$ such that $x \leq y$, and $x'_1, x'_2 : l(x)$: $$t(x, y)(x'_1 \vee x'_2) = t(x, y)(x'_1) \vee t(x, y)(x'_2)$$.
      \end{itemize}

      We write $\mathrm{SemiLFibration}(L)$ for the set of semilattice fibrations on $L$.

      This is equivalent to the above category-theoretic definition. $l$ corresponds to the semilattice fibration functor's action on objects, and $t$ to its action on morphisms (itself a functor). The first two conditions ensure that $(l, t)$ is a functor, while the third ensures that $t(x, y, \cdot)$ is a semilattice morphism.

      Intuitively, the first condition states that transporting a value $x'$ when not changing its dependency $x$ doesn't change $x'$; the second condition states dependency updates aren't path-dependent; and the third states transports distribute over joins.

      The total space $\mathrm{SemiLTot}(L, l, t)$ is then a semilattice whose elements are pairs $(x, x')$ where $x : L$ and $x' : l(x)$. The join is defined as $(x, x') \vee (y, y') := (x \vee y, t(x, x \vee y)(x') \vee t(y, x \vee y)(y'))$. This is equivalent to the above category-theoretic definition, proving it is a semilattice.

    \section{Semilattice graphs}

      Semilattice fibrations can be composed iteratively in a linear fashion. What is more general is to compose them in a directed acyclic graph structure.

      Suppose we have a well-ordered, countable set $K$ of keys, and a set $V$ of values. For each key $k \in K$, 

    \section{Composing semilattice fibrations}

      Semilattice fibrations can be composed iteratively. Composing finitely is straightforward, while composing infinitely requires more work.

      Suppose we have:

      \begin{align*}
        L_0 &: \mathrm{Semilattice} & \\
        (l_n, t_n) &: \mathrm{SemiLFibration}(L_{n-1}) & \text{ for natural $n \geq 1$ } \\
        L_n &:= \mathrm{SemiLTot}(L_{n-1}, l_n, t_n)    & \text{ for natural $n \geq 1$ }
      \end{align*}

      We now wish to define $L_\omega$ to be a semilattice that is, morally speaking, the limit of $L_n$. This lattice's elements are infinite lists $x_1, x_2, \ldots$ satisfying $x_n : l_n(x_{<n})$ for each natural $n \geq 1$, where $x_{<n} = x_1, \ldots, x_{n-1} : L_{n-1}$. The joins are defined as follows:

      $$(\overline{x} \vee \overline{y})_n := t_n(x_{<n}, x_{<n} \vee y_{<n})(x_n) \vee_{l_n(x_{<n} \vee y_{<n})} t_n(y_{<n}, x_{<n} \vee y_{<n})(y_n)$$

      \begin{theorem}
        $L_{\omega}$ is a semilattice.
      \end{theorem}
      \begin{proof}

        To verify that $L_\omega$ is a semilattice, we must verify three properties:
        \begin{itemize}
          \item \emph{Commutativity}:
            \begin{align*}
              (\overline{x} \vee \overline{y})_n
              &= t_n(x_{<n}, x_{<n} \vee y_{<n})(x_n) \vee_{l_n(x_{<n} \vee y_{<n})} t_n(y_{<n}, x_{<n} \vee y_{<n})(y_n) \\
              &= t_n(y_{<n}, y_{<n} \vee x_{<n})(y_n) \vee_{l_n(y_{<n} \vee x_{<n})} t_n(x_{<n}, y_{<n} \vee x_{<n})(x_n) \\
              &= (\overline{y} \vee \overline{x})_n
            \end{align*}
          \item \emph{Associativity}:
            \begin{align*}
              & ((\overline{x} \vee \overline{y}) \vee \overline{z})_n \\
              =& t_n(x_{<n} \vee y_{<n}, x_{<n} \vee y_{<n} \vee z_{<n})((\overline{x} \vee \overline{y})_n) \vee_{l_n(x_{<n} \vee y_{<n} \vee z_{<n})}
                 t_n(z_{<n}, x_{<n} \vee y_{<n} \vee z_{<n})(z_n)  \\
              =& t_n(x_{<n} \vee y_{<n}, x_{<n} \vee y_{<n} \vee z_{<n})(
                          t_n(x_{<n}, x_{<n} \vee y_{<n})(x_n) \vee_{l_n(x_{<n} \vee y_{<n})}
                          t_n(y_{<n}, x_{<n} \vee y_{<n})(y_n)) \\
               &\vee_{l_n(x_{<n} \vee y_{<n} \vee z_{<n})} t_n(z_{<n}, x_{<n} \vee y_{<n} \vee z_{<n})(z_n)  \\
              =& (t_n(x_{<n} \vee y_{<n}, x_{<n} \vee y_{<n} \vee z_{<n}) \circ t_n(x_{<n}, x_{<n} \vee y_{<n}))(x_n) \vee_{l_n(x_{<n} \vee y_{<n} \vee z_{<n})} \\
               & (t_n(x_{<n} \vee y_{<n}, x_{<n} \vee y_{<n} \vee z_{<n}) \circ t_n(y_{<n}, x_{<n} \vee y_{<n}))(y_n) \vee_{l_n(x_{<n} \vee y_{<n} \vee z_{<n})} \\
               & t_n(z_{<n}, x_{<n} \vee y_{<n} \vee z_{<n})(z_n)  \\
              =& t_n(x_{<n}, x_{<n} \vee y_{<n} \vee z_{<n})(x_n) \vee_{l_n(x_{<n} \vee y_{<n} \vee z_{<n})} \\
               & t_n(y_{<n}, x_{<n} \vee y_{<n} \vee z_{<n})(y_n) \vee_{l_n(x_{<n} \vee y_{<n} \vee z_{<n})} \\
               & t_n(z_{<n}, x_{<n} \vee y_{<n} \vee z_{<n})(z_n)  \\
              =& t_n(x_{<n}, x_{<n} \vee y_{<n} \vee z_{<n})(x_n) \vee_{l_n(x_{<n} \vee y_{<n} \vee z_{<n})} \\
               & (t_n(y_{<n} \vee z_{<n}, x_{<n} \vee y_{<n} \vee z_{<n}) \circ t_n(y_{<n}, y_{<n} \vee z_{<n}))(y_n) \vee_{l_n(x_{<n} \vee y_{<n} \vee z_{<n})} \\
               & (t_n(y_{<n} \vee z_{<n}, x_{<n} \vee y_{<n} \vee z_{<n}) \circ t_n(z_{<n}, y_{<n} \vee z_{<n}))(z_n)  \\
              =& t_n(x_{<n}, x_{<n} \vee y_{<n} \vee z_{<n})(x_n) \vee_{l_n(x_{<n} \vee y_{<n} \vee z_{<n})} \\
               & t_n(y_{<n} \vee z_{<n}, x_{<n} \vee y_{<n} \vee z_{<n})(
                          t_n(y_{<n}, y_{<n} \vee z_{<n})(y_n) \vee_{l_n(y_{<n} \vee z_{<n})}
                          t_n(z_{<n}, y_{<n} \vee z_{<n})(z_n)) \\
              =& t_n(x_{<n}, x_{<n} \vee y_{<n} \vee z_{<n})(x_n) \vee_{l_n(x_{<n} \vee y_{<n} \vee z_{<n})} \\
               & t_n(y_{<n} \vee z_{<n}, x_{<n} \vee y_{<n} \vee z_{<n})((\overline{y} \vee \overline{z})_n) \\
              =& (\overline{x} \vee (\overline{y} \vee \overline{z}))_n
            \end{align*}
          \item \emph{Idempotence}:
            \begin{align*}
              (\overline{x} \vee \overline{x})_n &= t_n(x_{<n}, x_{<n} \vee x_{<n})(x_n) \vee_{l_n(x_{<n} \vee x_{<n}))} t_n(x_{<n}, x_{<n} \vee x_{<n})(x_n) \\
              &= t_n(x_{<n}, x_{<n})(x_n) \\
              &= x_n
            \end{align*}
        \end{itemize}
      \end{proof}

      We can further stack to higher ordinals than $\omega$. Let $n_u \geq 1$ be the number of universes. Assume we have:

      \begin{align*}
        L_{0, 0} &: \mathrm{Semilattice} & \\
        (l_{u, n}, t_{u, n}) &: \mathrm{SemiLFibration}(L_{u, n-1}) & \text{for natural $0 \leq u < n_u$, $n \geq 1$ } \\
        L_{u, n} &:= \mathrm{SemiLTot}(L_{u, n-1}, l_{u, n}, t_{u, n})    & \text{ for natural $0 \leq u < n_u$, $n \geq 1$ } \\
        L_{u, 0} &:= L_{u-1, \omega} & \text{ for natural $1 \leq u < n_u$ }
      \end{align*}

      where $L_{u, \omega}$ is constructed the same way as $L_{\omega}$ above, as the limit of $L_{u, n}$ as $n$ approaches $\infty$. Inductively, each $L_{u, n}$ is a semilattice for natural $0 \leq u < n_u, n \geq 0$; this lattice's ordinal corresponds to $u \omega + n$. While in theory it may be possible to reach higher ordinals $\omega^2$ and beyond, we consider this beyond the scope of this work.

    \section{Semilattice Graphs}

      In terms of implementation, it is by default infeasible to construct an element of the semilattice $L_{0, \omega}$, as it is infinite. To ameliorate this problem, we consider semilattices with a bottom element $\bot$ satisfying $\bot \vee x = x$ for each $x$ in the semilattice. We write $\mathrm{Semilattice}_\bot$ for the type of semilattices with $\bot$; $\mathrm{SemiLFibration}_\bot(L)$ for the type of bottom-preserving semilattice fibrations on $L : \mathrm{Semilattice}_\bot$; and $\mathrm{SemiLTot}_\bot(L, l, t) : \mathrm{Semilattice}_\bot$ for the total space the semilattice $L : \mathrm{Semilattice}_\bot$ and fibration $(l, t) : \mathrm{SemiLFibration}_\bot$.

      Using the above definition of semilattice universes, we can now explicitly represent elements of $L_{u, n}$ for which only a finite number of entries are not $\bot$. Joins are straightforward since if one element of the joined universe structure is $\bot$, the join equals the other element. The bottoms of the total space and $\omega$ total spaces are trivial, containing only $\bot$.

      Let $K$ be a countable, well-ordered set of keys whose ordinal is no more than $\omega^2$. Let $V$ be a set of values. Since keys correspond with ordinals in $\omega^2$, we can write any key as $u\omega + n$; let $u(k)$ and $n(k)$ be $u$ and $n$ in this term respectively. We define a \emph{semilattice graph} to consist of:
      \begin{align*}
        L_{0, 0} &:= \{ \bot \} \\
        (l_k, t_k) &: \mathrm{SemiLFibration}_\bot(L_{u(k), n(k)-1}) & \text{ for each $k \in K$ } \\
        l_k(x) \subseteq V &\text{ for each $k \in K$ and $x \in L_{u(k), n(k)}$} \\
        L_{u, n} &:= \mathrm{SemiLTot}_\bot(L_{u, n-1}, l_{u, n}, t_{u, n})    & \text{ for natural $0 \leq u < n_u$, $n \geq 1$ } \\
        L_{u, 0} &:= L_{u-1, \omega} & \text{ for natural $1 \leq u < n_u$ }
      \end{align*}

      Note that each key corresponds to a semilattice fibration from a multiverse containing lower keys. We can re-interpret each element of the lattice $L_{u(k), n(k)}$ as mappings from keys less no higher than $k$ (written as $K_{\leq k}$) to $V$, which doesn't affect the semilattice structure. Accordingly, $l_k$ and $t_k$ are re-interpreted as taking mappings as arguments.

      It is somewhat unwieldy to define $l_k$ and $t_k$ in implementation, since they depend on a possibly-infinite mapping. To make the representation tractable, we assume $l_k$ and $t_k$ are \emph{continuous} in the sense that each of their outputs depend on only the values of a finite number of keys.

      A continuous function $g_k : (K_{<k} \rightarrow V) \rightarrow S$, with $K_{<k}, V, S$ treated as having a discrete topology and the $K_{<k} \rightarrow V$ space treated as having the product topology, is continuous if for each $s \in S$, the preimage $f_k^{-1}(s)$ is open. An subset $F \subseteq K_{<k} \rightarrow V$ is open if it is a union of basis sets of the form $\{ f \in K_{<k} \rightarrow V | f(k_1) = v_1 \wedge \ldots \wedge f(k_n) = v_n \}$. A given $f : K_{<k} \rightarrow V$ in this set is, accordingly, in at least one of these bases. Any $f'$ that agrees with $f$ on the keys of this basis therefore has the same value $g(f') = g(f)$. We say the elements of the smallest such basis set are the dependencies of $g$ in $f$.

      Let $d_k^\star : 2^{K_{<k}}$ be a set of lattice dependencies, and $d_{k}(v) : 2^{K_{<k}}$ be a set of value dependencies for $v \in V$. Both dependency sets must only include keys less than $k$. We say $(l_k, d_k)$ have dependencies according to $d_{k, l}, d_{k, v}$ if:

      \begin{itemize}
        \item $\mathrm{Elems}(l_k(m))$ only depends on $m(k')$ for each $k' \in d_k^\star$.
        \item $\bot_{l_k(m})$ only depends on $m(k')$ for each $k' \in d_k^\star$.
        \item $t_k(m, m')(v)$ only depends on $m(k')$ and $m'(k')$ for each $k' \in d_k^\star \cup d_k(v)$.
        \item $x \vee_{l_k(m)} y$ only depends on $m(k')$ for each $k' \in d_k^\star \cup d_k(x) \cup d_k(y)$.
      \end{itemize}

      To ensure dependency sets don't grow unexpectedly, we require:
      \begin{itemize}
        \item $d_k(t_k(m, m')(v)) \subseteq d_k(v)$.
        \item $d_k(x \vee_{l_k(m)} y) \subseteq d_k(x) \cup d_k(y)$.
        \item $d_k(\bot_{l_k(m)}) \subseteq d_k^\star$.
      \end{itemize}


    \section{Semilattice Graphs}

        A common problem in networked computing is dependency managament. A software package may depend on a set of other packages, each of which depends on another set of packages. We consider extending semilattices to semilattice graphs to handle dependency management.
        
        A \emph{semilattice graph} consists of:

        \begin{itemize}
            \item A set of keys $K$.
            \item A nonempty set of values $V$.
            \item A dependency function $d : K \times V \rightarrow 2^K$ that maps each key and value to a finite set of keys, its dependencies.
            \item A transport function $t : \Pi (k : K, v : V), (d(k, v) \rightarrow V)^2 \rightarrow V$ that transports a value when its dependencies change.
            \item A join function $j : \Pi (k : K, v_1, v_2 : V), (d(k, v_1) \cup d(k, v_2) \rightarrow V) \rightarrow V$ that maps two values to a value given their dependencies.
        \end{itemize}

        We write $v_1 \vee_{k, g} v_2 := j(k, v_1, v_2, g|_{d(k, v_1) \cup d(k, v_2)})$, and $t_{k, g_1, g_2}(v) := t(k, v, g_1|_{d(k, v)}, g_2|_{d(k, v)})$.

        The semilattice graph must satisfy the following properties:

        \begin{itemize}
            \item $d$ must be well-founded; that is, there are no infinite dependency chains $k_1, k_2, \ldots \in K$ and $v_1, v_2, \ldots \in V$ such that $k_{i+1} \in d(k_i, v_i)$ for $i = 1, 2, \ldots$
            \item $t$ must not grow dependencies: for $k : K, v : V, g_1, g_2 : d(k, v) \rightarrow V$, we must have $d(k, t(k, v, g_1, g_2)) \subseteq d(k, v)$.
            \item $j$ must not grow dependencies; for any $k : K, v_1 : V, v_2: V, g: d(k, v_1) \cup d(k, v_2) \rightarrow V$, we must have $d(k, j(k, v_1, v_2, g)) \subseteq d(k, v_1) \cup d(k, v_2)$.
            \item $t$ must be an identity when dependencies are equal: for $k : K, v : V, g : d(k, v) \rightarrow V$, we must have $t(k, v, g, g) = v$.
            \item $t$ must not be path-dependent: for $k : K, v : V, g_1, g_2, g_3: d(k, v) \rightarrow V$, we must have $t_{k, g_2, g_3} \circ t_{k, g_1, g_2} = t_{k, g_1, g_3}$.
            \item $t$ must distribute over $j$: for $k : K, v_1, v_2 : V, g_1, g_2 : d(k, v_1) \cup d(k, v_2) \rightarrow V$, we must have $t_{k, g_1, g_2}(v_1 \vee_{k, g_1} v_2) = t_{k, g_1, g_2}(v_1) \vee_{k, g_2} t_{k, g_1, g_2}(v_2)$.
            \item For any $k: K, g : K \rightarrow V$, we must have that $\vee_{k, g}$ is commutative, associative, and idempotent.
        \end{itemize}

        It is now possible to define a semilattice over a semilattice graph. This lattice's items are functions $g : K \rightarrow V$ (called \emph{mappings}). The join operation is defined as follows:

        $$ (g_1 \vee g_2)(k) := t_{k, g_1, g_1 \vee g_2}(g_1(k)) \vee_{k, g_1 \vee g_2} t_{k, g_2, g_1 \vee g_2}(g_2(k))$$.

        We must now show that $\vee$ is commutative, associative, and idempotent, inductively. For any natural $n$ we say $\vee$ (applied to semilattice graphs) forms a semilattice up to $n$ if, for any $k : K$ whose longest dependency chain is no longer than $n$, we have:

        \begin{itemize}
          \item \emph{Commutative}: $(g_1 \vee g_2)(k) = (g_2 \vee g_1)(k)$
          \item \emph{Associative}: $((g_1 \vee g_2) \vee g_3)(k) = (g_1 \vee (g_2 \vee g_3))(k)$
          \item \emph{Idempotent}: $(g \vee g)(k) = g(k)$
        \end{itemize}

        We note that if $\vee$ forms a semilattice up to all $n$, then it forms a semilattice. We now note that if $\vee$ forms a semilattice up to $n$, then it forms a semilattice up to $n+1$, inductively:

        \begin{itemize}
          \item \emph{Commutative}:
            \begin{align*}
              (g_1 \vee g_2)(k) &= t_{k, g_1, g_1 \vee g_2}(g_1(k)) \vee_{k, g_1 \vee g_2} t_{k, g_2, g_1 \vee g_2}(g_2(k)) \\
              &= t_{k, g_2, g_2 \vee g_1}(g_2(k)) \vee_{k, g_2 \vee g_1} t_{k, g_1, g_2 \vee g_1}(g_1(k)) \\
              &= (g_2 \vee g_1)(k)
            \end{align*}
          \item \emph{Associative}:
            \begin{align*}
              &((g_1 \vee g_2) \vee g_3))(k) \\
              =& t_{k, g_1 \vee g_2, g_1 \vee g_2 \vee g_3}((g_1 \vee g_2)(k)) \vee_{k, g_1 \vee g_2 \vee g_3} t_{k, g_3, g_1 \vee g_2 \vee g_3}(g_3(k)) \\
              =& t_{k, g_1 \vee g_2, g_1 \vee g_2 \vee g_3}\left(t_{k, g_1, g_1 \vee g_2}(g_1(k)) \vee_{k, g_1 \vee g_2} t_{k, g_2, g_1 \vee g_2}(g_2(k))\right) \\
               & \vee_{k, g_1 \vee g_2 \vee g_3} t_{k, g_3, g_1 \vee g_2 \vee g_3}(g_3(k)) \\
              =& t_{k, g_1 \vee g_2, g_1 \vee g_2 \vee g_3}(t_{k, g_1, g_1 \vee g_2}(g_1(k))) \vee_{k, g_1 \vee g_2 \vee g_3} \\
               & t_{k, g_1 \vee g_2, g_1 \vee g_2 \vee g_3}(t_{k, g_2, g_1 \vee g_2}(g_2(k))) \vee_{k, g_1 \vee g_2 \vee g_3} \\
               & t_{k, g_3, g_1 \vee g_2 \vee g_3}(g_3(k)) \\
              =& t_{k, g_1, g_1 \vee g_2 \vee g_3}(g_1(k)) \vee_{k, g_1 \vee g_2 \vee g_3}
                 t_{k, g_2, g_1 \vee g_2 \vee g_3}(g_2(k)) \vee_{k, g_1 \vee g_2 \vee g_3}
                 t_{k, g_3, g_1 \vee g_2 \vee g_3}(g_3(k)) \\
              =& t_{k, g_1, g_1 \vee g_2 \vee g_3}(g_1(k)) \vee_{k, g_1 \vee g_2 \vee g_3} \\
               & t_{k, g_2 \vee g_3, g_1 \vee g_2 \vee g_3}(t_{k, g_2, g_2 \vee g_3}(g_2(k))) \vee_{k, g_1 \vee g_2 \vee g_3} \\
               & t_{k, g_2 \vee g_3, g_1 \vee g_2 \vee g_3}(t_{k, g_3, g_2 \vee g_3}(g_3(k))) \\
              =& t_{k, g_1, g_1 \vee g_2 \vee g_3}(g_1(k)) \vee_{k, g_1 \vee g_2 \vee g_3} \\
               & t_{k, g_2 \vee g_3}(t_{k, g_2, g_2 \vee g_3}(g_2(k)) \vee_{k, g_2 \vee g_3} t_{k, g_3, g_2 \vee g_3}(g_3(k))) \\
              =& t_{k, g_1, g_1 \vee g_2 \vee g_3}(g_1(k)) \vee_{k, g_1 \vee g_2 \vee g_3} t_{k, g_2 \vee g_3}((g_2 \vee g_3)(k)) \\
              =& (g_1 \vee (g_2 \vee g_3))(k)
            \end{align*}
          \item \emph{Idempotent}:
            \begin{align*}
              (g \vee g)(k) &= t_{k, g, g \vee g}(g(k)) \vee_{k, g \vee g} t_{k, g, g \vee g}(g(k)) \\
              &= t_{k, g, g}(g(k)) \vee_{k, g} t_{k, g, g}(g(k)) \\
              &= g(k)
            \end{align*}
        \end{itemize}

\end{document}
