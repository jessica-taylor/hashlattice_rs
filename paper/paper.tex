
\documentclass{article}

\usepackage{amsthm}
\usepackage{amsmath}
\usepackage{amssymb}
\newtheorem{theorem}{Theorem}
\newtheorem{lemma}{Lemma}

\title{Semilattice Graphs for Distributed Databases}
\author{Jessica Taylor}
\date{\today}

\begin{document}
    \maketitle
    
    \section{Introduction}
        A \emph{semilattice} is a set $L$ with a binary operation $\vee$ (``join'') that is commutative ($x \vee y = y \vee x$), associative ($(x \vee y) \vee z = x \vee (y \vee z)$), and idempotent ($x \vee x = x$).

        We write $x \leq y$ to mean $y = x \vee y$.

        In databases, lattices are useful for defining a notion of a most updated value. For example, with $L$ as a set of sets of messages and $\vee = \cup$, the lattice represents sets of messages that are updated to include more messages as they come in.

    \section{Semilattices as a category}

      We discuss category theoretic language for semilattices to derive a notion of dependency between semilattices. In the next section, we drop the category-theoretic language, so it is not strictly necessary to understand category theory to understand the result, although it helps with the derivation.

      A semilattice can be interpreted as a category whose objects are the members of $L$, and where a morphism exists between $x$ and $y$ if and only if $x \leq y$; in this case there must be exactly one morphism of this form denoted $x \leq y$. To be a semilattice, the category must have coproducts (corresponding to joins).

      Note that if $x \leq y$ and $y \leq x$, then $x = x \vee y = y$.

      Semilattices form a category $\mathrm{Semilattice}$.  The objects are semilattices and the morphisms are coproduct-preserving functors between semilattices. A functor between two semilattices $L_1, L_2$ consists of a function $f: L_1 \rightarrow L_2$ mapping elements of $L_1$ to elements of $L_2$ that preserves joins in the sense that for $x_1 : L_1, x_2 : L_2$, $f(x_1 \vee x_2) = f(x_1) \vee f(x_2)$. From this condition it follows that if $x \leq y$, then $f(x) \leq f(x) \vee f(y) = f(x \vee y) = f(y)$. The $\mathrm{Semilattice}$ category has products in the obvious way.

      A semilattice fibration on a semilattice $L$ is a functor $g : L \rightarrow \mathrm{Semilattice}$, which assigns to each element $x$ of $L$ a semilattice $g(x)$, and assigns to each ordering $x \leq y$ a semilattice functor $g(x \leq y)$ from $g(x)$ to $g(y)$.

      The total space's elements consist of pairs $(x, x')$ where $x : L$ and $x' : g(x)$. We consider $(x, x') \leq (y, y')$ when $x \leq y$ and $g(x \leq y)(x') \leq y'$. The total space is a semilattice, as shown:

      \begin{theorem}
        The total space of a semilattice fibration $g$ is a semilattice.
      \end{theorem}
      \begin{proof}
        Consider two elements of the total space, $(x, x'), (y, y')$. Their join is defined as $(x, x') \vee (y, y') := (x \vee y', g(x \leq x \vee y)(x') \vee g(y \leq x \vee y)(y'))$. We must show that this join is commutative, associative, and idempotent.
        \begin{itemize}
          \item \emph{Commutative}:
            \begin{align*}
              (x, x') \vee (y, y') &= (x \vee y, g(x \leq x \vee y)(x') \vee g(y \leq x \vee y)(y')) \\
              &= (y \vee x, g(y \leq y \vee x)(y') \vee g(x \leq y \vee x)(x')) \\
              &= (y, y') \vee (x, x')
            \end{align*}
          \item \emph{Associative}:
            \begin{align*} &((x, x') \vee (y, y')) \vee (z, z')  \\
              =& (x \vee y \vee z, g(x \vee y \leq x \vee y \vee z)(g(x \leq x \vee y)(x') \vee g(y \leq x \vee y)(y')) \vee g(z \leq x \vee y \vee z)(z')) \\
              =& (x \vee y \vee z, g(x \leq x \vee y \vee z)(x') \vee g(y \leq x \vee y \vee z)(y') \vee g(z \leq x \vee y \vee z)(z') \\
              =& (x \vee y \vee z, g(x \leq x \vee y \vee z)(x') \vee g(y \vee z \leq x \vee y \vee z)(g(y \leq y \vee z)(y') \vee g(z \leq y \vee z)(z')) \\
              =& (x, x') \vee ((y, y') \vee (z, z'))
            \end{align*}
          \item \emph{Idempotent}: 
            \begin{align*}
              (x, x') \vee (x, x') &= (x \vee x, g(x \leq x \vee x)(x') \vee g(x \leq x \vee x)(x')) \\
              &= (x, g(x \leq x)(x')) \\
              &= (x, x')
            \end{align*}
        \end{itemize}
      \end{proof}

    \section{De-categorizing semilattice fibrations}

      To define a semilattice fibration more directly, we can remove the category-theoretic language, achieving a more direct presentation.

      A semilattice fibration consists of a function $g_l : L \rightarrow \mathrm{Semilattice}$, assigning to each element of $L$ a semilattice, and a function $g_t : \Pi (x, y : L, x \leq y, x' : g(x)), g(y)$ (a ``transport function''), satisfying:

      \begin{itemize}
        \item For any $x : L, x' : g_l(x)$: $$g_t(x, x, x') = x'$$.
        \item For any $x, y, z : L$ such that $x \leq y \leq z$, and $x' : g_l(x)$: $$g_t(y, z, g_t(x, y, x')) = g_t(x, z, x')$$.
        \item For any $x, y : L$ such that $x \leq y$, and $x'_1, x'_2 : g_l(x)$: $$g_t(x, y, x'_1 \vee x'_2) = g_t(x, y, x'_1) \vee g_t(x, y, x'_2)$$.
      \end{itemize}

      This is equivalent to the above category-theoretic definition. $g_l$ corresponds to the semilattice fibration functor's action on objects, and $g_t$ to its action on morphisms (itself a functor). The first two conditions ensure that $(g_l, g_t)$ is a functor, while the third ensures that $g_t(x, y, \cdot)$ is a semilattice functor.

      Intuitively, the first condition states that transporting a value $x'$ when not changing its dependency $x$ doesn't change $x'$; the second condition states dependency updates aren't path-dependent; and the third states transports distribute over joins.

      The total space is then a semilattice whose elements are pairs $(x, x')$ where $x : L$ and $x' : g_l(x)$. The join is defined as $(x, x') \vee (y, y') := (x \vee y, g_t(x, x \vee y, x') \vee g_t(y, x \vee y, y'))$. This is equivalent to the above category-theoretic definition, proving it is a semilattice.

    \section{Semilattice Graphs}

        A common problem in networked computing is dependency managament. A software package may depend on a set of other packages, each of which depends on another set of packages. We consider extending semilattices to semilattice graphs to handle dependency management.
        
        A \emph{semilattice graph} consists of:

        \begin{itemize}
            \item A set of keys $K$.
            \item A nonempty set of values $V$.
            \item A dependency function $d : K \times V \rightarrow 2^K$ that maps each key and value to a finite set of keys, its dependencies.
            \item A transport function $t : \Pi (k : K, v : V), (d(k, v) \rightarrow V)^2 \rightarrow V$ that transports a value when its dependencies change.
            \item A join function $j : \Pi (k : K, v_1, v_2 : V), (d(k, v_1) \cup d(k, v_2) \rightarrow V) \rightarrow V$ that maps two values to a value given their dependencies.
        \end{itemize}

        We write $v_1 \vee_{k, g} v_2 := j(k, v_1, v_2, g|_{d(k, v_1) \cup d(k, v_2)})$, and $t_{k, g_1, g_2}(v) := t(k, v, g_1|_{d(k, v)}, g_2|_{d(k, v)})$.

        The semilattice graph must satisfy the following properties:

        \begin{itemize}
            \item $d$ must be well-founded; that is, there are no infinite dependency chains $k_1, k_2, \ldots \in K$ and $v_1, v_2, \ldots \in V$ such that $k_{i+1} \in d(k_i, v_i)$ for $i = 1, 2, \ldots$
            \item $t$ must not grow dependencies: for $k : K, v : V, g_1, g_2 : d(k, v) \rightarrow V$, we must have $d(k, t(k, v, g_1, g_2)) \subseteq d(k, v)$.
            \item $j$ must not grow dependencies; that is, for any $k : K, v_1 : V, v_2: V, g: (d(k, v_1) \cup d(k, v_2)) \rightarrow V$, we must have $d(k, j(k, v_1, v_2, g)) \subseteq d(k, v_1) \cup d(k, v_2)$.
            \item $t$ must be an identity when dependencies are equal: for $k : K, v : V, g : d(k, v) \rightarrow V$, we must have $t(k, v, g, g) = v$.
            \item $t$ must not be path-dependent: for $k : K, v : V, g_1, g_2, g_3 : d(k, v) \rightarrow V$, we must have $t(k, t(k, v, g_1, g_2), g_2, g_3) = t(k, v, g_1, g_3)$.
            \item $t$ must distribute over $j$: for $k : K, v_1, v_2 : V, g_1, g_2 : d(k, v_1) \cup d(k, v_2) \rightarrow V$, we must have $t_{k, g_1, g_2}(v_1 \vee_{k, g_1} v_2) = t_{k, g_1, g_2}(v_1) \vee_{k, g_2} t_{k, g_1, g_2}(v_2)$.
            \item For any $k: K, g : K \rightarrow V$, we must have that $\vee_{k, g}$ is commutative, associative, and idempotent.
        \end{itemize}

        It is now possible to define a semilattice over a semilattice graph. This lattice's items are functions $g : K \rightarrow V$ (called \emph{mappings}). The join operation is defined as follows:

        $$ (g_1 \vee g_2)(k) := t_{k, g_1, (g_1 \vee g_2)}(g_1(k)) \vee_{k, (g_1 \vee g_2)} t_{k, g_2, (g_1 \vee g_2)}(g_2(k))$$.

        We must now show that $\vee$ is commutative, associative, and idempotent, inductively. For any natural $n$ we say $\vee$ (applied to semilattice graphs) forms a semilattice up to $n$ if, for any $k : K$ whose longest dependency chain is no longer than $n$, we have:

        \begin{itemize}
          \item $(g_1 \vee g_2)(k) = (g_2 \vee g_1)(k)$
          \item $((g_1 \vee g_2) \vee g_3)(k) = (g_1 \vee (g_2 \vee g_3))(k)$
          \item $(g \vee g)(k) = g(k)$
        \end{itemize}

        We note that if $\vee$ forms a semilattice up to all $n$, then it forms a semilattice. We now note that if $\vee$ forms a semilattice up to $n$, then it forms a semilattice up to $n+1$, inductively:

        \begin{itemize}
          \item \emph{Commutative}:
            \begin{align*}
              (g_1 \vee g_2)(k) &= t_{k, g_1, (g_1 \vee g_2)}(g_1(k)) \vee_{k, (g_1 \vee g_2)} t_{k, g_2, (g_1 \vee g_2)}(g_2(k)) \\
              &= t_{k, g_2, (g_2 \vee g_1)}(g_2(k)) \vee_{k, (g_2 \vee g_1)} t_{k, g_1, (g_2 \vee g_1)}(g_1(k)) \\
              &= (g_2 \vee g_1)(k)
            \end{align*}
          \item \emph{Associative}:
            \begin{align*}
              &((g_1 \vee g_2) \vee g_3))(k) \\
              =& t_{k, (g_1 \vee g_2), (g_1 \vee g_2 \vee g_3)}((g_1 \vee g_2)(k)) \vee_{k, (g_1 \vee g_2 \vee g_3)} t_{k, g_3, (g_1 \vee g_2 \vee g_3)}(g_3(k)) \\
              =& t_{k, (g_1 \vee g_2), (g_1 \vee g_2 \vee g_3)}(t_{k, g_1, (g_1 \vee g_2)}(g_1(k)) \vee_{k, (g_1 \vee g_2)} t_{k, g_2, (g_1 \vee g_2)}(g_2(k))) \\
               & \vee_{k, (g_1 \vee g_2 \vee g_3)} t_{k, g_3, (g_1 \vee g_2 \vee g_3)}(g_3(k)) \\
              =& t_{k, (g_1 \vee g_2), (g_1 \vee g_2 \vee g_3)}(t_{k, g_1, (g_1 \vee g_2)}(g_1(k))) \vee_{k, (g_1 \vee g_2 \vee g_3)} \\
               & t_{k, (g_1 \vee g_2), (g_1 \vee g_2 \vee g_3)}(t_{k, g_2, (g_1 \vee g_2)}(g_2(k))) \vee_{k, (g_1 \vee g_2 \vee g_3)} \\
               & t_{k, g_3, (g_1 \vee g_2 \vee g_3)}(g_3(k)) \\
              =& t_{k, g_1, (g_1 \vee g_2 \vee g_3)}(g_1(k)) \vee_{k, (g_1 \vee g_2 \vee g_3)}
                 t_{k, g_2, (g_1 \vee g_2 \vee g_3)}(g_2(k)) \vee_{k, (g_1 \vee g_2 \vee g_3)}
                 t_{k, g_3, (g_1 \vee g_2 \vee g_3)}(g_3(k)) \\
              =& t_{k, g_1, (g_1 \vee g_2 \vee g_3)}(g_1(k)) \vee_{k, (g_1 \vee g_2 \vee g_3)} \\
               & t_{k, (g_2 \vee g_3), (g_1 \vee g_2 \vee g_3)}(t_{k, g_2, (g_2 \vee g_3)}(g_2(k))) \vee_{k, (g_1 \vee g_2 \vee g_3)} \\
               & t_{k, (g_2 \vee g_3), (g_1 \vee g_2 \vee g_3)}(t_{k, g_3, (g_2 \vee g_3)}(g_3(k))) \\
              =& t_{k, g_1, (g_1 \vee g_2 \vee g_3)}(g_1(k)) \vee_{k, (g_1 \vee g_2 \vee g_3)} \\
               & t_{k, (g_2 \vee g_3)}(t_{k, g_2, (g_2 \vee g_3)}(g_2(k)) \vee_{k, (g_2 \vee g_3)} t_{k, g_3, (g_2 \vee g_3)}(g_3(k))) \\
              =& t_{k, g_1, (g_1 \vee g_2 \vee g_3)}(g_1(k)) \vee_{k, (g_1 \vee g_2 \vee g_3)} t_{k, (g_2 \vee g_3)}((g_2 \vee g_3)(k)) \\
              =& (g_1 \vee (g_2 \vee g_3))(k)
            \end{align*}
          \item \emph{Idempotent}:
            \begin{align*}
              (g \vee g)(k) &= t_{k, g, (g \vee g)}(g(k)) \vee_{k, (g \vee g)} t_{k, g, (g \vee g)}(g(k)) \\
              &= t_{k, g, g}(g(k)) \vee_{k, g} t_{k, g, g}(g(k)) \\
              &= g(k)
            \end{align*}
        \end{itemize}

    \section{Autofills}

        Mappings may be infinite in general, and even finite ones contain a large number of keys. It is therefore computationally infeasible to store a mapping as a set of $K \times V$ pairs. Accordingly, we define \emph{autofills} to fill in lattice values automatically from a set of other values.


        % Define $d^*(k, g) := \{k\} \cup \bigcup_{k' \in d(k, g(k))} d^*(k', g)$, the transitive closure of $d$, where $g$ is a mapping. The recursion is well-founded (and $d^*(k, g)$ is finite) because $d$ is well-founded.

        % Note that $d^*(k, g_1 \vee g_2) \subseteq d^*(k, g_1) \cup d^*(k, g_2)$.

        An \emph{autofill} for a semilattice graph consists of a dependency function $\hat{d}: K \rightarrow 2^V$ and an autofill function $a : \forall (k:K), \hat{d}(k) \rightarrow V$ that satisfy:

        \begin{itemize}
          \item
            For each $k \in K, v \in V$, $\hat{d}(k) \subseteq d(k, v)$.
          \item
            $a(k, g_1) \vee_{k, g_1 \vee g_2} a(k, g_2) = a(k, g_1 \vee g_2)$; that is, the join of two autofill values on two mappings must be an autofill value on the joined mapping.
        \end{itemize}

        where we write $a(k, g) := a(k, g|_{\hat{d}(k)})$ for brevity.


        A mapping $g : K \rightarrow V$ is an \emph{autofill mapping} if:
        
        \begin{itemize}
          \item $g(k) = a(k, g)$ for all but a finite number of $k$ values.
          \item $g(k) \vee_{k, g} a(k, g) = g(k)$ for all $k \in K$.
        \end{itemize}


        An autofill mapping can be represented as a finite set of $K \times V$ pairs; the remaining values are filled in automatically.

        \begin{theorem}
          Let $g_1, g_2 : K \rightarrow V$ be autofill mappings. Then $g_1 \vee g_2$ is an autofill mapping.
        \end{theorem}

        \begin{proof}

          Whenever $g_1(k) = a(k, g_1)$ and $g_2(k) = a(k, g_2)$ (which is true for all but a finite number of $k$ values):
           $$(g_1 \vee g_2)(k) = g_1(k) \vee_{k, (g_1 \vee g_2)} g_2(k) = a(k, g_1) \vee_{k, g_1 \vee g_2} a(k, g_2) = a(k, g_1 \vee g_2)$$

          Additionally, for each $k \in K$:
          \begin{align*}
            (g_1 \vee g_2)(k) \vee_{k, (g_1 \vee g_2)} a(k, g) 
            &= (g_1(k) \vee_{k, (g_1 \vee g_2)} g_2(k)) \vee_{k, (g_1 \vee g_2)} a(k, g) \\
            &= g_1(k) \vee_{k, (g_1 \vee g_2)} (g_2(k) \vee_{k, (g_1 \vee g_2)} a(k, g)) \\
            &= g_1(k) \vee_{k, (g_1 \vee g_2)} g_2(k) \\
            &= (g_1 \vee g_2)(k)
          \end{align*}
        \end{proof}

        For an autofill mapping $g$, we can define the \emph{concrete dependencies} of a key $k \in K$, which are a set of non-autofill keys of $g$ that are sufficient to derive $k$'s dependencies:

        $$c(k, g) := \bigcup_{k' \in d(k, g(k))} \begin{cases}
          \{k'\} & \text{if } g(k') \neq a(k, g) \\
          c(k', g) & \text{otherwise}
        \end{cases}  $$


        Merging two mappings can be done recursively. Given mappings $g_1$ and $g_2$ and a key $k$, we compute the values $g_1(k), g_2(k)$ and their dependency sets $d(k, g_1(k)), d(k, g_2(k))$. We recursively merge $g_1$ and $g_2$ on each key in either dependency set, caching intermediate values to avoid repetated recursive calls. Finally, we compute the join and cache it in the database.

        We may consider merging a mapping with a singleton mapping, which specifies the value of a single key and otherwise contains only default values. In this case, merging will change this value and all its direct or indirect dependencies. Merging with a singleton is similar to a ``database insert'', updating the value of a given key.

\end{document}
